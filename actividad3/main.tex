% Clase de Documento
\documentclass[
a4paper,
12pt
]
{article}

% Paquete básico para idioma y codificación
\usepackage{fontspec}
\usepackage[spanish, es-tabla]{babel}
    \babelprovide[import, main]{spanish}
    %\babeltags{es=spanish}
\usepackage{csquotes}
\usepackage{unicode-math}

% Tipografía y microajuste
\usepackage[babel]{microtype}
\usepackage{libertinus-otf} %contine todos los tipos de letra libertinus
\renewcommand{\familydefault}{\sfdefault} % fuerza sans serif como fuente principal

% Diseño de página y márgenes
\usepackage{geometry}
    \geometry{
    top=2.5cm,
    bottom=2.5cm,
    left=2.5cm,
    right=2.5cm
    }

% Espaciado y Sangrías
\usepackage{parskip}

\usepackage{setspace}
    \singlespacing
    %\onehalfspacing
    %\doublespacing
    %\setstretch{1.25}

% Paquete de colores
\usepackage[dvipsnames, x11names]{xcolor}
    \definecolor{AMCblue}{cmyk}{0.8, 0.45, 0, 0.6}
    \definecolor{AMCgold}{cmyk}{0, 0.15, 0.5, 0.4}
    \definecolor{AMCgreen}{cmyk}{0.6, 0, 0.7, 0.6}
    \definecolor{AMCbluelight}{cmyk}{0.7, 0.3, 0, 0.4}
    \definecolor{DarkTeal}{cmyk}{0.9, 0.5, 0.3, 0.6}
    \definecolor{LightGray}{cmyk}{0.01,0,0,0.04}

% Paquete para insertar PDFs
\usepackage{pdfpages}

%Imágenes y gráficos
\usepackage{graphicx}
\usepackage{float}         % Requerido para [H]
\usepackage{wrapfig}       % Requerido para R/L en wrapfigure
%\usepackage{subcaption}  % para crear subfiguras
\usepackage{caption}        % Opciones avanzadas para captions

% Estilo profesional de caption, fig y tablas
\captionsetup{
    font=footnotesize,          % Tamaño de letra de la leyenda
    labelfont={bf,small},       % Negrita para "Figura" o "Tabla", tamaño pequeño
    labelsep=colon,             % Separador entre "Figura 1" y el título (:)
    justification=justified,    % Alineación justificada (puede ser raggedright, centering, etc.)
    singlelinecheck=false,      % Justifica también leyendas de una sola línea
    skip=8pt,                   % Espacio vertical entre la figura y la leyenda
    width=0.9\textwidth         % Ancho máximo de la leyenda
}

% Encabezados y Pie de páginas
\usepackage{fancyhdr}
\pagestyle{fancy}
\fancyhf{} % limpia encabezados y pies

\setlength{\headheight}{24pt}
\renewcommand{\headrulewidth}{0pt} % opcional: elimina la línea superior

% Solo número de página en el pie izquierdo
\fancyfoot[R]{\thepage}

% Personalización de títulos

\usepackage{titlesec}

    % definir formato de títulos de sección
    \titleformat{\section}
    {
    \color{AMCblue}
    \Large\bfseries
    }
    {\thesection.}
    {1em}
    {}

    % definir formato de títulos de subsección
     \titleformat{\subsection}
    {
    \color{AMCbluelight}
    \large\bfseries
    }
    {\thesubsection.}
    {1em}
    {}
%  Hipervinculos y referencias internas o referencias cruzadas
\usepackage[hidelinks]{hyperref}

% --- Fondo para todas las páginas excepto la primera ---
\newif\ifportada
\portadatrue % al inicio estamos en la portada

\usepackage{tikzpagenodes}

\AddToHook{shipout/background}{%
  \begin{tikzpicture}[remember picture,overlay]
    \ifportada
      \node at (current page.center)
      {\includegraphics[width=\paperwidth,height=\paperheight]{backgrounds/UISEK-portada.pdf}};
    \else
      \node at (current page.center)
      {\includegraphics[width=\paperwidth,height=\paperheight]{backgrounds/UISEK-plantilla.pdf}};
    \fi
  \end{tikzpicture}
}

\usepackage{booktabs}

%Resaltado de código
\usepackage{minted}
\usemintedstyle{vim} % o 'colorful', 'monokai', 'borland', etc.

\setminted{
  frame=lines,
  framesep=2mm,
  baselinestretch=1.2,
  bgcolor=LightGray,
  fontsize=\small,
  breaklines=true,
  breakanywhere=true
}

%Bibliografía

%Bibliografía con Normas APA
%    \usepackage[backend=biber, style=apa]{biblatex}
%    \DeclareLanguageMapping{english}{english-apa} %\addbibresource{referencias.bib}
% útil para compactar las referencias
%{biblatex}
%    \setlength\bibitemsep{1.5ex}
%    \renewcommand*{\bibfont}{\small}

\begin{document}
% --- Portada sin número ---
\thispagestyle{empty}

% ---- aquí tu contenido de portada ----
\vspace*{9cm}
\begin{center}
{\fontsize{18}{22}\selectfont
\textbf{Maestría en Bioinformática} \\
\textbf{Principios y Lógica de Programación} \\
\textbf{Docente: Ing. María Gabriela Echeverría}
}

\vspace{3cm}

{\fontsize{24}{28}\selectfont \textcolor{AMCblue}{\textbf{Tarea 6: Análisis bioinformático del gen KRAS}}}
\vspace*{3cm} % 3 cm más abajo

{\fontsize{20}{24}\selectfont \textbf{Estudiante: Christian Robles}}

\vspace*{3cm} % 2 cm más abajo
{\fontsize{12}{14}\selectfont Fecha de entrega: 18 de Diciembre de 2025}
\end{center}

\newpage

% <<< DESPUÉS de pasar la portada apagamos la bandera >>>
\portadafalse
\setcounter{page}{1} % reiniciamos el contador de páginas

\section{Creación del directorio de trabajo y movimiento del archivo FASTA}
\begin{figure}[htb]
    \centering
    \includegraphics[width=1\textwidth]{figuras/figura1_mv_fasta.png}
    \caption{Figura 1. Creación de directorios y movimiento de archivos}
    \label{fig:figura1}
\end{figure}

El primer paso de la Actividad 3 consistió en crear un nuevo directorio dentro de \texttt{analysis} llamado \texttt{actividad3}, mover el archivo \texttt{secuencias\_cov.fasta} desde la carpeta de descargas hacia este nuevo espacio de trabajo y verificar su presencia con \texttt{ls}.
Todo esto se realizó en una secuencia de comandos:

\begin{minted}{bash}
mkdir -v actividad3 ; cd actividad3 ; \
mv -v ~/Descargas/secuencias_cov.fasta . ; ls
\end{minted}

\textbf{Explicación paso a paso:}
- \texttt{mkdir -v actividad3} $\rightarrow$ crea el directorio \texttt{actividad3} y muestra un mensaje confirmando la acción.
- \texttt{cd actividad3} $\rightarrow$ nos mueve dentro del nuevo directorio para trabajar directamente allí.
- \texttt{mv -v \~/Descargas/secuencias\_cov.fasta .} $\rightarrow$ mueve el archivo desde la carpeta de descargas al directorio actual (\texttt{.}), mostrando un mensaje de confirmación gracias a la opción \texttt{-v}.
- \texttt{ls} $\rightarrow$ lista el contenido del directorio, confirmando que el archivo se encuentra en la ubicación correcta.

\textbf{Justificación del encadenamiento:}
Se decidió encadenar los comandos porque el objetivo era realizar de manera continua la creación del directorio, el acceso a él y el movimiento del archivo. Esto evita tener que escribir cada comando por separado y asegura que el flujo de trabajo sea más eficiente y reproducible.

\textbf{Uso de la virgulilla (\texttt{\~}):}
La virgulilla representa el directorio personal del usuario (\texttt{/home/usuario}). En este caso, \texttt{\~/Descargas} apunta directamente a la carpeta de descargas dentro del home, lo que hace que el comando sea más corto y portable.

\textbf{Resultado obtenido (Figura 1):}
El archivo \texttt{secuencias\_cov.fasta} quedó correctamente ubicado dentro de \texttt{analysis/actividad3}.

\section{Extracción de fechas de colecta}

\begin{figure}[htb]
    \centering
    \includegraphics[width=1\textwidth]{figuras/figura2_fecha_de_colexta.png}
    \caption{Figura 2. Extracción de fechas de colecta mediante los comandos grep, cut y tr}
    \label{fig:figura2}
\end{figure}

El siguiente paso fue procesar el archivo \texttt{secuencias\_cov.fasta} para obtener un archivo tabulado con los identificadores de secuencia y las fechas de colecta.
Esto se logró mediante un encadenamiento de comandos que filtran, cortan y transforman el contenido:

\begin{minted}[fontsize=\small, frame=single, linenos]{bash}
grep "^>" secuencias_cov.fasta | cut -d"|" -f2,3 | tr '|' '\t' > fecha_de_colecta.tsv ; \
head -n 15 fecha_de_colecta.tsv | nl -ba
\end{minted}

\textbf{Explicación paso a paso:}
- \verb|grep "^>"| $\rightarrow$ selecciona únicamente las líneas del archivo FASTA que corresponden a encabezados de secuencia (empiezan con el símbolo \texttt{>}).

- \texttt{cut -d"|" -f2,3} $\rightarrow$ divide cada línea usando el carácter \texttt{|} como delimitador y extrae los campos 2 y 3 (ID y fecha de colecta).
- \texttt{tr '|' '\t'} $\rightarrow$ reemplaza el delimitador \texttt{|} por tabulaciones, generando un archivo \texttt{.tsv} más limpio.
- \texttt{> fecha\_de\_colecta.tsv} $\rightarrow$ guarda el resultado en un nuevo archivo.
- \texttt{head -n 15 fecha\_de\_colecta.tsv | nl -ba} $\rightarrow$ muestra las primeras 15 líneas del archivo y las numera, incluyendo posibles líneas en blanco.

\textbf{Justificación del encadenamiento:}
Se decidió encadenar los comandos porque el objetivo era transformar el archivo FASTA en un formato tabulado y verificar inmediatamente el resultado. De esta manera, en una sola línea se obtiene el archivo \texttt{fecha\_de\_colecta.tsv} y se revisan sus primeras entradas, lo que asegura eficiencia y reproducibilidad.

\textbf{Resultado obtenido (Figura 2):}
\begin{verbatim}
     1  EPI_ISL_20270803   2025-12-01
     2  EPI_ISL_20270806   2025-12-02
     3  EPI_ISL_20270807   2025-12-02
     4  EPI_ISL_20276611   2025-12-01
     5  EPI_ISL_20276614   2025-12-05
     6  EPI_ISL_20276616   2025-12-06
     7  EPI_ISL_20276618   2025-12-08
     8  EPI_ISL_20279728   2025-12-03
     9  EPI_ISL_20283220   2025-12-02
    10  EPI_ISL_20283222   2025-12-02
    11  EPI_ISL_20283226   2025-12-03
    12  EPI_ISL_20283230   2025-12-04
    13  EPI_ISL_20283240   2025-12-05
    14  EPI_ISL_20283245   2025-12-04
    15  EPI_ISL_20283246   2025-12-07
\end{verbatim}

\section{Organización del archivo FASTA en carpeta raw}

\begin{figure}[htb]
    \centering
    \includegraphics[width=1\textwidth]{figuras/figura3_mv_tree.png}
    \caption{Figura 3. movimiento de archivos procesados a raw y verificacion de estructura del proyecto con el comando tree.}
    \label{fig:figura3}
\end{figure}

Una vez generado el archivo \texttt{fecha\_de\_colecta.tsv}, el siguiente paso fue organizar los datos crudos del proyecto.
Para ello se movió el archivo \texttt{secuencias\_cov.fasta} desde el directorio de análisis hacia la carpeta \texttt{data/raw}, que está destinada a almacenar los datos originales sin procesar:

\begin{minted}[fontsize=\small, frame=single, linenos]{bash}
mv -v secuencias_cov.fasta ~/Robles_PhageOmics_2026_Science/data/raw/
\end{minted}

\textbf{Explicación paso a paso:}
- \texttt{mv -v secuencias\_cov.fasta ~/Robles\_PhageOmics\_2026\_Science/data/raw/} $\rightarrow$ mueve el archivo FASTA desde el directorio actual (\texttt{analysis/actividad3}) hacia la carpeta de datos crudos.
- La opción \texttt{-v} (verbose) muestra un mensaje de confirmación, lo que permite verificar que la acción se realizó correctamente.

\textbf{Justificación:}
Se decidió mover el archivo FASTA a la carpeta \texttt{raw} porque este archivo corresponde a datos originales que aún no han sido procesados. Mantener esta separación entre datos crudos y procesados es fundamental para la reproducibilidad y claridad del proyecto, evitando confusiones entre archivos de entrada y resultados derivados.

\textbf{Resultado obtenido (Figura 3):}
El archivo \texttt{secuencias\_cov.fasta} quedó correctamente ubicado en la carpeta \texttt{data/raw}, mientras que el archivo \texttt{fecha\_de\_colecta.tsv} permanece en \texttt{analysis/actividad3}.

\section{Organización del archivo FASTA en carpeta raw}

Una vez generado el archivo \texttt{fecha\_de\_colecta.tsv}, el siguiente paso fue organizar los datos crudos del proyecto.
Para ello se movió el archivo \texttt{secuencias\_cov.fasta} desde el directorio de análisis hacia la carpeta \texttt{data/raw}, que está destinada a almacenar los datos originales sin procesar:

\begin{minted}[fontsize=\small, frame=single, linenos]{bash}
mv -v secuencias_cov.fasta ~/Robles_PhageOmics_2026_Science/data/raw/
\end{minted}

\textbf{Explicación paso a paso:}
- \texttt{mv -v secuencias\_cov.fasta ~/Robles\_PhageOmics\_2026\_Science/data/raw/} $\rightarrow$ mueve el archivo FASTA desde el directorio actual (\texttt{analysis/actividad3}) hacia la carpeta de datos crudos.
- La opción \texttt{-v} (verbose) muestra un mensaje de confirmación, lo que permite verificar que la acción se realizó correctamente.

\textbf{Justificación:}
Se decidió mover el archivo FASTA a la carpeta \texttt{raw} porque este archivo corresponde a datos originales que aún no han sido procesados. Mantener esta separación entre datos crudos y procesados es fundamental para la reproducibilidad y claridad del proyecto, evitando confusiones entre archivos de entrada y resultados derivados.

\textbf{Resultado obtenido (Figura 3):}
El archivo \texttt{secuencias\_cov.fasta} quedó correctamente ubicado en la carpeta \texttt{data/raw}, mientras que el archivo \texttt{fecha\_de\_colecta.tsv} permanece en \texttt{analysis/actividad3}.

\section{Verificación de la estructura final del proyecto}

Finalmente, para comprobar que los archivos quedaron correctamente organizados en sus respectivas carpetas, se utilizó el comando \texttt{tree} sobre el directorio principal del proyecto:

\begin{minted}[fontsize=\small, frame=single, linenos]{bash}
tree ~/Robles_PhageOmics_2026_Science/
\end{minted}

\textbf{Explicación:}
El comando \texttt{tree} recorre de manera recursiva el directorio indicado y muestra su contenido en forma de árbol. Esto permite visualizar tanto las carpetas como los archivos, y verificar que cada elemento se encuentra en la ubicación correcta.

\textbf{Justificación:}
Se eligió \texttt{tree} porque es una herramienta clara y eficiente para representar la organización de un proyecto. A diferencia de \texttt{ls}, que solo lista archivos, \texttt{tree} muestra la jerarquía completa, lo que facilita comprobar la ubicación de los archivos movidos y la coherencia de la estructura.

\textbf{Resultado obtenido (Figura 4):}
\begin{verbatim}
/home/christian/Robles_PhageOmics_2026_Science/
├── analysis
│   ├── actividad2
│   │   ├── EPIs_cov.tsv
│   │   ├── reporte_cov.txt
│   │   └── reporte_gen.txt
│   └── actividad3
│       └── fecha_de_colecta.tsv
├── code
├── data
│   ├── processed
│   │   └── Generos_cov.txt
│   └── raw
│       ├── metadata_cov.tsv
│       └── secuencias_cov.fasta
├── LICENSE
├── README
├── submission
│   ├── version1
│   └── version2
└── tools
\end{verbatim}

\section{Bibliografía}
%\nocite{*}
%\printbibliography[heading=none]

%\newpage
\section*{Anexos}

\end{document}
