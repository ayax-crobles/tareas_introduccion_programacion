% Clase de Documento
\documentclass[
a4paper,
12pt,
%draft % para no renderizar imágenes hasta que todo esté correcto
]
{article}

%\special{pdf:minorversion=7}

% Paquete básico para idioma y codificación
\usepackage{fontspec}
\usepackage[spanish, es-tabla]{babel}
    \babelprovide[import, main]{spanish}
    %\babeltags{es=spanish}
\usepackage{unicode-math}

% Tipografía y microajuste
\usepackage[babel]{microtype}
\usepackage{libertinus-otf} %contine todos los tipos de letra libertinus
%\setsansfont{Libertinus Sans} %En este caso como no va a ser impreso vamos a usar sans
\renewcommand{\familydefault}{\sfdefault} % fuerza sans serif como fuente principal
%\setmonofont{Libertinus Mono}

% Diseño de página y márgenes
\usepackage{geometry}
    \geometry{
    top=2.5cm,
    bottom=2.5cm,
    left=2.5cm,
    right=2.5cm
    }

% Espaciado y Sangrías
\usepackage{parskip}

\usepackage{setspace}
    \singlespacing
    %\onehalfspacing
    %\doublespacing
    %\setstretch{1.25}

% Paquete de colores
\usepackage[dvipsnames, x11names]{xcolor}
    \definecolor{AMCblue}{cmyk}{0.8, 0.45, 0, 0.6}
    \definecolor{AMCgold}{cmyk}{0, 0.15, 0.5, 0.4}
    \definecolor{AMCgreen}{cmyk}{0.6, 0, 0.7, 0.6}
    \definecolor{AMCbluelight}{cmyk}{0.7, 0.3, 0, 0.4}
    \definecolor{DarkTeal}{cmyk}{0.9, 0.5, 0.3, 0.6}
    \definecolor{LightGray}{cmyk}{0.01,0,0,0.04}

% Paquete para insertar PDFs
\usepackage{pdfpages}

%Imágenes y gráficos
\usepackage{graphicx}
\usepackage{float}         % Requerido para [H]
\usepackage{wrapfig}       % Requerido para R/L en wrapfigure
%\usepackage{subcaption}  % para crear subfiguras
\usepackage{caption}        % Opciones avanzadas para captions

% Estilo profesional de caption, fig y tablas
\captionsetup{
    font=footnotesize,          % Tamaño de letra de la leyenda
    labelfont={bf,small},       % Negrita para "Figura" o "Tabla", tamaño pequeño
    labelsep=colon,             % Separador entre "Figura 1" y el título (:)
    justification=justified,    % Alineación justificada (puede ser raggedright, centering, etc.)
    singlelinecheck=false,      % Justifica también leyendas de una sola línea
    skip=8pt,                   % Espacio vertical entre la figura y la leyenda
    width=0.9\textwidth         % Ancho máximo de la leyenda
}

% Encabezados y Pie de páginas
\usepackage{fancyhdr}
\pagestyle{fancy}
\fancyhf{} % limpia encabezados y pies

\setlength{\headheight}{24pt}
\renewcommand{\headrulewidth}{0pt} % opcional: elimina la línea superior

% Solo número de página en el pie izquierdo
\fancyfoot[R]{\thepage}

% Personalización de títulos

\usepackage{titlesec}

    % definir formato de títulos de sección
    \titleformat{\section}
    {
    \color{AMCblue}
    \Large\bfseries
    }
    {\thesection.}
    {1em}
    {}

    % definir formato de títulos de subsección
     \titleformat{\subsection}
    {
    \color{AMCbluelight}
    \large\bfseries
    }
    {\thesubsection.}
    {1em}
    {}
%  Hipervinculos y referencias internas o referencias cruzadas
\usepackage[hidelinks]{hyperref}
% --- Fondo para todas las páginas excepto la primera ---
\newif\ifportada
\portadatrue % al inicio estamos en la portada

\usepackage{tikzpagenodes}

\AddToHook{shipout/background}{%
  \begin{tikzpicture}[remember picture,overlay]
    \ifportada
      \node at (current page.center)
      {\includegraphics[width=\paperwidth,height=\paperheight]{backgrounds/UISEK-portada.pdf}};
    \else
      \node at (current page.center)
      {\includegraphics[width=\paperwidth,height=\paperheight]{backgrounds/UISEK-plantilla.pdf}};
    \fi
  \end{tikzpicture}
}

\usepackage{booktabs}

%Resaltado de código
\usepackage{minted}
\usemintedstyle{emacs} % o 'colorful', 'monokai', 'borland', etc.

\setminted{
  %frame=lines,
  %framesep=2mm,
  baselinestretch=1.2,
  bgcolor=LightGray,
  fontsize=\small,
  breaklines=true,
  breakanywhere=true
}


\usepackage{csquotes} %cambiamos a este lugar cs quotes por si está antes de minted da problemas

%Bibliografía

%Bibliografía con Normas APA
\usepackage[backend=biber, style=apa]{biblatex}
\DeclareLanguageMapping{english}{english-apa}
\addbibresource{referencias.bib}

% útil para compactar las referencias
%{biblatex}
%    \setlength\bibitemsep{1.5ex}
%    \renewcommand*{\bibfont}{\small}

\begin{document}
% --- Portada sin número ---
\thispagestyle{empty}

% ---- aquí tu contenido de portada ----
\vspace*{9cm}
\begin{center}
{\fontsize{18}{22}\selectfont
\textbf{Maestría en Bioinformática} \\
\textbf{Principios y Lógica de Programación} \\
\textbf{Docente: Ing. María Gabriela Echeverría}
}

\vspace{3cm}

{\fontsize{24}{28}\selectfont \textcolor{AMCblue}{\textbf{Actividad 3: Extracción de datos y ordenamiento de secuencias en arhivos multifasta}}}
\vspace*{3cm} % 3 cm más abajo

{\fontsize{20}{24}\selectfont \textbf{Estudiante: Christian Robles}}

\vspace*{3cm} % 2 cm más abajo
{\fontsize{12}{14}\selectfont Fecha de entrega: 20 de Enero de 2026}
\end{center}

\newpage

% <<< DESPUÉS de pasar la portada apagamos la bandera >>>
\portadafalse
\setcounter{page}{1} % reiniciamos el contador de páginas

\section{Introducción}

En esta tercera actividad trabajamos con el archivo \texttt{secuencias\_cov.fasta}, descargado desde el aula virtual, con el propósito de practicar la extracción de información en los encabezados de cada secuencia. A partir de estos datos se generan reportes organizados que incluyen el número de acceso (\texttt{Accession ID}) y la fecha de colecta, ordenados de manera sistemática. La tarea se realizó en \textbf{Debian 13 Trixie}, utilizando el entorno de escritorio \textbf{XFCE}, \textbf{xfce4-terminal} y el gestor de archivos \textbf{Thunar}.

\section{Creación del directorio de trabajo y movimiento del archivo FASTA}
\begin{figure}[htb]
    \centering
    \includegraphics[width=1\textwidth]{figuras/figura1_mv_fasta.png}
    \caption{Figura 1. Creación de directorios y movimiento de archivos}
    \label{fig:figura1}
\end{figure}

El primer paso de la Actividad 3 consistió en crear un nuevo directorio dentro de \texttt{analysis} llamado \texttt{actividad3}, mover el archivo \texttt{secuencias\_cov.fasta} desde la carpeta de descargas hacia este nuevo directorio y verificar su presencia con \texttt{ls}.
Todo esto se realizó con los comandos:

\begin{minted}{bash}
mkdir -v actividad3
cd actividad3
mv -v ~/Descargas/secuencias_cov.fasta .
ls
\end{minted}

\textbf{Comandos utilizados:}
\begin{itemize}
  \item \texttt{mkdir -v actividad3} $\rightarrow$ crea el directorio \texttt{actividad3} y muestra un mensaje de confirmación.
  \item \texttt{cd actividad3} $\rightarrow$ nos mueve dentro del nuevo directorio para trabajar directamente allí.
  \item \texttt{mv -v \~/Descargas/secuencias\_cov.fasta .} $\rightarrow$ mueve el archivo desde la carpeta de descargas al directorio actual (\texttt{.}), mostrando un mensaje de lo realizado gracias a la opción \texttt{-v} (verbose).
  \item \texttt{ls} $\rightarrow$ lista el contenido del directorio, confirmando que el archivo se encuentra en el directorio actual.
  \item El archivo \texttt{secuencias\_cov.fasta} quedó ubicado dentro de \texttt{analysis/actividad3}.
\end{itemize}

\section{Extracción de fechas de colecta}

\begin{figure}[htb]
    \centering
    \includegraphics[width=1\textwidth]{figuras/figura2_fecha_de_colexta.png}
    \caption{Figura 2. Extracción de fechas de colecta mediante los comandos grep, cut y tr}
    \label{fig:figura2}
\end{figure}

El siguiente paso fue procesar el archivo \texttt{secuencias\_cov.fasta} para obtener un archivo tabulado (para mejorar la visibilidad de los datos desde la terminal) con los Accession ID y las fechas de colecta.
Esto se logró mediante un encadenamiento de comandos que filtran, cortan y transforman el contenido:

\begin{minted}{bash}
grep "^>" secuencias_cov.fasta | cut -d"|" -f2,3 | tr '|' $'\t' | -k1n > fecha_de_colecta.tsv

head -n 15 fecha_de_colecta.tsv | nl -ba
\end{minted}

\textbf{Comandos utilizados:}
\begin{itemize}
  \item \verb|grep "^>"| $\rightarrow$ selecciona únicamente las líneas del archivo FASTA que corresponden a encabezados de secuencia (empiezan con el símbolo \texttt{>}).

  \item \texttt{cut -d"|" -f2,3} $\rightarrow$ divide cada línea usando el carácter \texttt{|} como delimitador y extrae los campos 2 y 3 (ID y fecha de colecta).
  \item \texttt{tr} $\rightarrow$ reemplaza el delimitador \texttt{|} por tabulaciones, generando un archivo \texttt{.tsv} lo que facilita su vizualización desde la terminal.
  \item \texttt{sort -k1n} $\rightarrow$ ordena el resultado por el primer campo con la opción \texttt{-k1} que corresponde a \texttt{Accesion ID} y lo hace por orden numerico gracias a la opción \texttt{-n}.
  \item \texttt{> fecha\_de\_colecta.tsv} $\rightarrow$ redirecciona la salida (el resultado) hacia un nuevo archivo.
  \item \texttt{head -n 15 fecha\_de\_colecta.tsv | nl -ba} $\rightarrow$ muestra las primeras 15 líneas del archivo y las numera (para mejorar la visibilidad), incluyendo posibles líneas en blanco.
\end{itemize}

Se decidió encadenar los comandos porque de esta manera, en una sola línea se obtiene el archivo \texttt{fecha\_de\_colecta.tsv} y se revisan sus primeras entradas, lo que asegura su eficiencia y reproducibilidad.

\section{Organización del archivo FASTA en carpeta raw}

\begin{figure}[htb]
    \centering
    \includegraphics[width=1\textwidth]{figuras/figura3_mv_tree.png}
    \caption{Figura 3. movimiento de archivos originales a raw y verificacion de estructura del proyecto con el comando tree.}
    \label{fig:figura3}
\end{figure}

Una vez generado el archivo \texttt{fecha\_de\_colecta.tsv}, el siguiente paso fue organizar los datos del proyecto.
Para ello se movió el archivo \texttt{secuencias\_cov.fasta} desde el directorio de análisis hacia la carpeta \texttt{data/raw}, que está destinada a almacenar los datos originales sin procesar:

\begin{minted}{bash}
mv -v secuencias_cov.fasta ~/Robles_PhageOmics_2026_Science/data/raw/
\end{minted}

%\textbf{Comandos Utilizados:}
%- \texttt{mv -v secuencias\_cov.fasta ~/Robles\_PhageOmics\_2026\_Science/data/raw/} $\rightarrow$ mueve el archivo FASTA desde el directorio actual (\texttt{analysis/actividad3}) hacia la carpeta de datos crudos.
%- La opción \texttt{-v} (verbose) muestra un mensaje de confirmación, lo que permite verificar que la acción se realizó correctamente.

%\textbf{Justificación:}
%Se decidió mover el archivo FASTA a la carpeta \texttt{raw} porque este archivo corresponde a datos originales que aún no han sido procesados. Mantener esta separación entre datos crudos y procesados es fundamental para la reproducibilidad y claridad del proyecto, evitando confusiones entre archivos de entrada y resultados derivados.

%\textbf{Resultado obtenido (Figura 3):}
%El archivo \texttt{secuencias\_cov.fasta} quedó correctamente ubicado en la carpeta \texttt{data/raw}, mientras que el archivo \texttt{fecha\_de\_colecta.tsv} permanece en \texttt{analysis/actividad3}.

\section{Verificación de la estructura final del proyecto}

Finalmente, para comprobar que los archivos quedaron correctamente organizados en sus respectivas carpetas, se utilizó el comando \texttt{tree} sobre el directorio principal del proyecto:

\begin{minted}{bash}
tree ~/Robles_PhageOmics_2026_Science/
\end{minted}

El comando \texttt{tree} recorre de manera recursiva el directorio indicado y muestra su contenido en forma de árbol. Esto permite visualizar tanto las carpetas como los archivos, y verificar que cada elemento se encuentra en la ubicación correcta.

\section{Conclusión}

La actividad permitió aplicar comandos de la terminal para extraer y organizar información contenida en los encabezados del archivo secuencias\_cov.fasta. Mediante los comandos grep, cut y tr. En conjunto, estas herramientas posibilitaron la generación de un reporte con datos clave como el número de acceso y la fecha de colecta, demostrando la eficiencia de la línea de comandos en la manipulación de información biológica.

\section{Bibliografía}
\nocite{*}
\printbibliography[heading=none]

%\newpage
%\section*{Anexos}

\end{document}
