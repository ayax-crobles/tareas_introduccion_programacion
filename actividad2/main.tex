% Clase de Documento
\documentclass[
a4paper,
12pt
]
{article}

% Paquete básico para idioma y codificación
\usepackage{fontspec}
\usepackage[spanish, es-tabla]{babel}
    \babelprovide[import, main]{spanish}
    %\babeltags{es=spanish}
\usepackage{unicode-math}

% Tipografía y microajuste
\usepackage[babel]{microtype}
\usepackage{libertinus-otf} %contine todos los tipos de letra libertinus
\renewcommand{\familydefault}{\sfdefault} % fuerza sans serif como fuente principal

% Diseño de página y márgenes
\usepackage{geometry}
    \geometry{
    top=2.5cm,
    bottom=2.5cm,
    left=2.5cm,
    right=2.5cm
    }

% Espaciado y Sangrías
\usepackage{parskip}

\usepackage{setspace}
    \singlespacing
    %\onehalfspacing
    %\doublespacing
    %\setstretch{1.25}

% Paquete de colores
\usepackage[dvipsnames, x11names]{xcolor}
    \definecolor{AMCblue}{cmyk}{0.8, 0.45, 0, 0.6}
    \definecolor{AMCgold}{cmyk}{0, 0.15, 0.5, 0.4}
    \definecolor{AMCgreen}{cmyk}{0.6, 0, 0.7, 0.6}
    \definecolor{AMCbluelight}{cmyk}{0.7, 0.3, 0, 0.4}
    \definecolor{DarkTeal}{cmyk}{0.9, 0.5, 0.3, 0.6}
    \definecolor{LightGray}{cmyk}{0.01,0,0,0.04}

% Paquete para insertar PDFs
\usepackage{pdfpages}

%Imágenes y gráficos
\usepackage{graphicx}
\usepackage{float}         % Requerido para [H]
\usepackage{wrapfig}       % Requerido para R/L en wrapfigure
%\usepackage{subcaption}  % para crear subfiguras
\usepackage{caption}        % Opciones avanzadas para captions

% Estilo profesional de caption, fig y tablas
\captionsetup{
    font=footnotesize,          % Tamaño de letra de la leyenda
    labelfont={bf,small},       % Negrita para "Figura" o "Tabla", tamaño pequeño
    labelsep=colon,             % Separador entre "Figura 1" y el título (:)
    justification=justified,    % Alineación justificada (puede ser raggedright, centering, etc.)
    singlelinecheck=false,      % Justifica también leyendas de una sola línea
    skip=8pt,                   % Espacio vertical entre la figura y la leyenda
    width=0.9\textwidth         % Ancho máximo de la leyenda
}

% Encabezados y Pie de páginas
\usepackage{fancyhdr}
\pagestyle{fancy}
\fancyhf{} % limpia encabezados y pies

\setlength{\headheight}{24pt}
\renewcommand{\headrulewidth}{0pt} % opcional: elimina la línea superior

% Solo número de página en el pie izquierdo
\fancyfoot[R]{\thepage}

% Personalización de títulos

\usepackage{titlesec}

    % definir formato de títulos de sección
    \titleformat{\section}
    {
    \color{AMCblue}
    \Large\bfseries
    }
    {\thesection.}
    {1em}
    {}

    % definir formato de títulos de subsección
     \titleformat{\subsection}
    {
    \color{AMCbluelight}
    \large\bfseries
    }
    {\thesubsection.}
    {1em}
    {}
%  Hipervinculos y referencias internas o referencias cruzadas
\usepackage[hidelinks]{hyperref}
% --- Fondo para todas las páginas excepto la primera ---
\newif\ifportada
\portadatrue % al inicio estamos en la portada

\usepackage{tikzpagenodes}

\AddToHook{shipout/background}{%
  \begin{tikzpicture}[remember picture,overlay]
    \ifportada
      \node at (current page.center)
      {\includegraphics[width=\paperwidth,height=\paperheight]{backgrounds/UISEK-portada.pdf}};
    \else
      \node at (current page.center)
      {\includegraphics[width=\paperwidth,height=\paperheight]{backgrounds/UISEK-plantilla.pdf}};
    \fi
  \end{tikzpicture}
}

\usepackage{booktabs}

%Resaltado de código
\usepackage{minted}
\usemintedstyle{emacs} % o 'colorful', 'monokai', 'borland', etc.

\setminted{
  %frame=lines,
  %framesep=2mm,
  baselinestretch=1.2,
  bgcolor=LightGray,
  fontsize=\small,
  breaklines=true,
  breakanywhere=true
}


\usepackage{csquotes} %cambiamos a este lugar cs quotes por si está antes de minted da problemas

%Bibliografía

%Bibliografía con Normas APA
\usepackage[backend=biber, style=apa]{biblatex}
\DeclareLanguageMapping{english}{english-apa}
\addbibresource{referencias.bib}
% útil para compactar las referencias
%{biblatex}
%    \setlength\bibitemsep{1.5ex}
%    \renewcommand*{\bibfont}{\small}

\begin{document}
% --- Portada sin número ---
\thispagestyle{empty}

% ---- aquí tu contenido de portada ----
\vspace*{9cm}
\begin{center}
{\fontsize{18}{22}\selectfont
\textbf{Maestría en Bioinformática} \\
\textbf{Principios y Lógica de Programación} \\
\textbf{Docente: Ing. María Gabriela Echeverría}
}

\vspace{3cm}

{\fontsize{24}{28}\selectfont \textcolor{AMCblue}{\textbf{Actividad 2: Procesamiento básico de metadatos en GNU/Linux}}}
\vspace*{3cm} % 3 cm más abajo

{\fontsize{20}{24}\selectfont \textbf{Estudiante: Christian Robles}}

\vspace*{3cm} % 2 cm más abajo
{\fontsize{12}{14}\selectfont Fecha de entrega: 20 de Enero de 2026}
\end{center}

\newpage

% <<< DESPUÉS de pasar la portada apagamos la bandera >>>
\portadafalse
\setcounter{page}{1} % reiniciamos el contador de páginas
\section{Introducción}
En esta segunda actividad trabajamos con el archivo \texttt{metadata\_cov.tsv}, descargado desde el aula virtual, con el objetivo principal de practicar la extracción de información relevante y la generación de reportes. La tarea se realizó en \textbf{Debian 13 Trixie}, utilizando el entorno de escritorio \textbf{XFCE}, \textbf{xfce4-terminal} y el gestor de archivos \textbf{Thunar}.

\section{Creación del directorio de trabajo y movimiento del archivo inicial}

El primer paso fue crear un nuevo directorio dentro de \texttt{analysis} llamado \texttt{actividad2}, entrar en él y mover el archivo \texttt{metadata\_cov.tsv} desde la carpeta de descargas hacia este nuevo espacio de trabajo.
Todo esto se realizó de la siguiente manera:

\begin{minted}{bash}
mkdir actividad2
cd actividad2
mv -v ~/Descargas/metadata_cov.tsv .
\end{minted}

\begin{figure}[htb]
    \centering
    \includegraphics[width=1\textwidth]{figuras/figura1_mv.png}
    \caption{Mover el archivo \texttt{metadata\_cov.tsv} al directorio de trabajo.}
    \label{fig:figura1}
\end{figure}

\textbf{Comandos utilizados:}
\begin{itemize}
  \item \texttt{mkdir actividad2} → crea la carpeta donde trabajaremos.
  \item \texttt{cd actividad2} → nos mueve directamente dentro de esa carpeta para no tener que escribir rutas largas después.
  \item \texttt{mv -v ~/Descargas/metadata\_cov.tsv .} → mueve el archivo desde la carpeta de descargas al directorio actual (\texttt{\textbf{.}}), mostrando un mensaje de confirmación gracias a la opción \texttt{-v} \textit{(verbose)}.
\end{itemize}


\textbf{Uso de la virgulilla:}
La virgulilla en Linux representa el directorio personal del usuario (\texttt{/home/usuario}). En este caso, \texttt{\~/Descargas} apunta directamente a la carpeta de descargas dentro de nuestro home. Usarla hace que los comandos sean más cortos y portables, ya que no necesitamos escribir la ruta completa.

\section{Extracción de IDs y creación del archivo \texttt{EPIs\_cov.tsv}}

El siguiente paso fue trabajar con el archivo \texttt{metadata\_cov.tsv} para obtener únicamente la columna de los identificadores de secuencias (Accession ID).
Esto se hizo con el comando:

\begin{minted}{bash}
cut -f2 metadata_cov.tsv > EPIs_cov.tsv
head -n 25 EPIs_cov.tsv | nl -ba
\end{minted}

\begin{figure}[htb]
    \centering
    \includegraphics[width=1\textwidth]{figuras/figura2_cut_head.png}
    \caption{Extracción de IDs con el comando \texttt{cut} y visualización de las primeras 25 líneas del archivo con el comando \texttt{head}}
    \label{fig:figura2}
\end{figure}


\textbf{Comandos utilizados:}
\begin{itemize}
  \item\texttt{cut -f2 metadata\_cov.tsv > EPIs\_cov.tsv} $\rightarrow$ con la opción \texttt{-f2} extrae la segunda columna del archivo original y la guarda en un nuevo archivo llamado \texttt{EPIs\_cov.tsv}.
  \item \texttt{head -n 25 EPIs\_cov.tsv | nl -ba} $\rightarrow$ muestra las primeras 25 líneas del archivo generado y las pasa por un pipe o tubería \texttt{(|)} hacia el comando \texttt{nl -ba} que numera todas las líneas, incluyendo las líneas en blanco si las hubiera, para facilitar la visualización.
\end{itemize}

%\vspace{2cm}
\textbf{Comandos \texttt{cut} y \texttt{head}:}
\begin{itemize}
  \item \texttt{cut} es ideal para extraer columnas específicas de archivos delimitados por tabulaciones, como los \texttt{.tsv}.
  \item\texttt{head} permite revisar solo una parte del archivo sin necesidad de abrirlo completo.
  \item \texttt{nl -ba} añade numeración a todas las líneas, lo que facilita la referencia y comprobación de los registros.
\end{itemize}

\section{Cantidad de muestras de covid en diciembre}

En este paso se trabajó con el archivo \texttt{EPIs\_cov.tsv} para calcular cuántas muestras de covid se registraron en diciembre.
Para ello se utilizó un bloque de comandos encadenados que primero añade un encabezado descriptivo y luego cuenta las líneas del archivo (excluyendo la cabecera) que corresponden al número total de muestras:

\begin{minted}{bash}
{ echo "La cantidad de muestras de covid en Diciembre es:" ;
  tail -n +2 EPIs_cov.tsv | wc -l ; } > reporte_cov.txt
  cat reporte_cov.txt
\end{minted}

\begin{figure}[htb]
    \centering
    \includegraphics[width=1\textwidth]{figuras/figura3_cantidad_muestras.png}
    \caption{Cantidad de muestras de covid en diciembre.}
    \label{fig:figura3}
\end{figure}

\textbf{Comandos utilizados:}
\begin{itemize}
  \item \texttt{echo "La cantidad de muestras de covid en Diciembre es:"} $\rightarrow$ imprime un mensaje que sirve como encabezado del reporte.
  \item \texttt{tail -n +2 EPIs\_cov.tsv | wc -l} $\rightarrow$ \textbf{tail -n +2} muestra a partir de la segunda fila el contenido del archivo de esta forma se elimina la primera línea (cabecera) y \textbf{wc-l} cuenta el número de líneas que corresponde al número de muestras. En casos donde la lógica es más compleja (muchos encabezados en el mismo archivo) se pueden usar herramientas mas potentes como \texttt{grep}, \texttt{awk} o \texttt{sed}.
  \item Todo el bloque entre llaves \texttt{\{ ... \}} se redirige con \texttt{>} al archivo \texttt{reporte\_cov.txt}.
  \item Finalmente, \texttt{cat reporte\_cov.txt} muestra el contenido del reporte en pantalla.
\end{itemize}

Se decidió agrupar los comandos dentro de llaves para generar un reporte completo en un solo paso: primero el encabezado, luego el conteo y redirigir la salida estándar del bloque de comandos dentro de las llaves hacia el archivo del reporte. Al encadenarlos, se asegura que el archivo \texttt{reporte\_cov.txt} quede listo con toda la información sin necesidad de editarlo manualmente después, o en varios pasos. Además, el comando \texttt{cat} al final permite verificar que el resultado es correcto.

\section{Distribución de muestras por género}

En este paso se trabajó nuevamente con el archivo \texttt{metadata\_cov.tsv}, parar obtener esta vez la información sobre el género de las muestras.
Se utilizó un bloque de comandos encadenados que primero genera el encabezado del reporte, extrae la columna de género y finalmente genera un reporte con los conteos:

\begin{minted}{bash}
cut -f8 metadata_cov.tsv > reporte_gen.txt
{ echo "La cantidad de muestras por género son:" ;
tail -n +2 reporte_gen.txt | sort | uniq -c ; } > Generos_cov.txt
cat Generos_cov.txt
\end{minted}

\begin{figure}[htb]
    \centering
    \includegraphics[width=1\textwidth]{figuras/figura4_generos.png}
    \caption{Ejemplo con múltiples opciones.png}
    \label{fig:figura4}
\end{figure}

%\vspace{2cm}
\textbf{Comandos utilizados:}
\begin{itemize}
  \item \texttt{cut -f8 metadata\_cov.tsv > reporte\_gen.txt} $\rightarrow$ extrae la octava columna del archivo original y la guarda en un archivo intermedio.
  \item Dentro de las llaves \texttt{\{ ... \}} se ejecutan dos acciones:
  \begin{enumerate}
    \item \texttt{echo "La cantidad de muestras por género son:"} $\rightarrow$ añade un encabezado descriptivo.
    \item \texttt{tail -n +2 reporte\_gen.txt | sort | uniq -c} $\rightarrow$ muestra a partir de la segunda fila el contenido del archivo de esta forma se elimina la primera línea que corresponde al encabezado, ordena los valores y cuenta cuántas veces aparece cada categoría de género Female, Male, y unknown.
  \end{enumerate}
  \item Todo este bloque se redirige al archivo \texttt{Generos\_cov.txt}.
  \item Finalmente, \texttt{cat Generos\_cov.txt} muestra el resultado en pantalla.
\end{itemize}

\textbf{El comando sort y unique:}
\begin{description}
  \item [sort | uniq -c :] permite agrupar y contar las categorías de manera sencilla. Primero, \texttt{sort} ordena las líneas del archivo, lo que es necesario para que \texttt{uniq} pueda identificar las líneas idénticas consecutivas. Luego, \texttt{uniq -c} cuenta cuántas veces aparece cada línea única, proporcionando un conteo claro de las muestras por género.
\end{description}

\section{Organización de archivos en subdirectorios}

Una vez generados los reportes, el siguiente paso fue organizar los archivos dentro de la estructura del proyecto.
Para ello se movieron los archivos hacia las carpetas \texttt{raw} y \texttt{processed}, según su tipo:

\begin{minted}{bash}
mv -v metadata_cov.tsv ~/Robles_PhageOmics_2026_Science/data/raw/
mv -v Generos_cov.txt ~/Robles_PhageOmics_2026_Science/data/processed/
\end{minted}

\begin{figure}[htb]
    \centering
    \includegraphics[width=0.8\textwidth]{figuras/figura5_raw_data.png}
    \caption{Movimiento de archivos a carpetas corrspondientes}
    \label{fig:figura5}
\end{figure}

%\textbf{Uso del comando \texttt{mv}:}
%\begin{itemize}
%  \item Mueve el archivo original de \texttt{metadata\_cov.tsv} a la carpeta de datos crudos (\texttt{raw}).
%  \item Mueve el archivo \texttt{Generos\_cov.txt} con el conteo por género a la carpeta de datos procesados.
%\end{itemize}

\section{Estructura final del proyecto}

Para verificar que todos los archivos quedaron correctamente organizados en sus carpetas, se utilizó el comando \texttt{tree} sobre el directorio principal del proyecto.
De esta forma se obtiene una vista jerárquica de la estructura completa:

\begin{minted}{bash}
tree ~/Robles_PhageOmics_2026_Science/
\end{minted}

\begin{figure}[htb]
    \centering
    \includegraphics[width=1\textwidth]{figuras/figura6_tree_final.png}
    \caption{Estructura final de directorios}
    \label{fig:figura6}
\end{figure}

El comando \texttt{tree} recorre de manera recursiva los directorios y muestra su contenido en forma de árbol. Esto permite visualizar tanto las carpetas como los archivos, y comprobar que cada elemento está en el lugar correcto.

%A diferencia de \texttt{ls}, que solo lista archivos, \texttt{tree} muestra la jerarquía completa, lo que facilita verificar la ubicación de los archivos movidos y la coherencia de la estructura.

\section{Conclusión}
La actividad permitió practicar la extracción de datos desde el archivo \texttt{metadata\_cov.tsv} y la generación de reportes claros y organizados. Mediante el uso de comandos encadenados, redirecciones de la salida estándar y pipes para filtrar información relevante, contar muestras y clasificarlas por género. %Esto refuerza la importancia de dominar estos comandos para producir reportes eficientes, documentar procesos y mantener orden en el análisis de datos.


\section{Bibliografía}
\nocite{*}
\printbibliography[heading=none]

%\newpage
%\section*{Anexos}

\end{document}
