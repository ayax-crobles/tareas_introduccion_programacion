% Clase de Documento
\documentclass[
a4paper,
12pt
]
{article}

% Paquete básico para idioma y codificación
\usepackage{fontspec}
\usepackage[spanish, es-tabla]{babel}
    \babelprovide[import, main]{spanish}
    %\babeltags{es=spanish}
\usepackage{unicode-math}

% Tipografía y microajuste
\usepackage[babel]{microtype}
\usepackage{libertinus-otf} %contine todos los tipos de letra libertinus
\renewcommand{\familydefault}{\sfdefault} % fuerza sans serif como fuente principal

% Diseño de página y márgenes
\usepackage{geometry}
    \geometry{
    top=2.5cm,
    bottom=2.5cm,
    left=2.5cm,
    right=2.5cm
    }

% Espaciado y Sangrías
\usepackage{parskip}

\usepackage{setspace}
    \singlespacing
    %\onehalfspacing
    %\doublespacing
    %\setstretch{1.25}

% Paquete de colores
\usepackage[dvipsnames, x11names]{xcolor}
    \definecolor{AMCblue}{cmyk}{0.8, 0.45, 0, 0.6}
    \definecolor{AMCgold}{cmyk}{0, 0.15, 0.5, 0.4}
    \definecolor{AMCgreen}{cmyk}{0.6, 0, 0.7, 0.6}
    \definecolor{AMCbluelight}{cmyk}{0.7, 0.3, 0, 0.4}
    \definecolor{DarkTeal}{cmyk}{0.9, 0.5, 0.3, 0.6}
    \definecolor{LightGray}{cmyk}{0.01,0,0,0.04}

% Paquete para insertar PDFs
\usepackage{pdfpages}

%Imágenes y gráficos
\usepackage{graphicx}
\usepackage{float}         % Requerido para [H]
\usepackage{wrapfig}       % Requerido para R/L en wrapfigure
%\usepackage{subcaption}  % para crear subfiguras
\usepackage{caption}        % Opciones avanzadas para captions

% Estilo profesional de caption, fig y tablas
\captionsetup{
    font=footnotesize,          % Tamaño de letra de la leyenda
    labelfont={bf,small},       % Negrita para "Figura" o "Tabla", tamaño pequeño
    labelsep=colon,             % Separador entre "Figura 1" y el título (:)
    justification=justified,    % Alineación justificada (puede ser raggedright, centering, etc.)
    singlelinecheck=false,      % Justifica también leyendas de una sola línea
    skip=8pt,                   % Espacio vertical entre la figura y la leyenda
    width=0.9\textwidth         % Ancho máximo de la leyenda
}

% Encabezados y Pie de páginas
\usepackage{fancyhdr}
\pagestyle{fancy}
\fancyhf{} % limpia encabezados y pies

\setlength{\headheight}{24pt}
\renewcommand{\headrulewidth}{0pt} % opcional: elimina la línea superior

% Solo número de página en el pie izquierdo
\fancyfoot[R]{\thepage}

% Personalización de títulos

\usepackage{titlesec}

    % definir formato de títulos de sección
    \titleformat{\section}
    {
    \color{AMCblue}
    \Large\bfseries
    }
    {\thesection.}
    {1em}
    {}

    % definir formato de títulos de subsección
     \titleformat{\subsection}
    {
    \color{AMCbluelight}
    \large\bfseries
    }
    {\thesubsection.}
    {1em}
    {}
%  Hipervinculos y referencias internas o referencias cruzadas
\usepackage[hidelinks]{hyperref}
% --- Fondo para todas las páginas excepto la primera ---
\newif\ifportada
\portadatrue % al inicio estamos en la portada

\usepackage{tikzpagenodes}

\AddToHook{shipout/background}{%
  \begin{tikzpicture}[remember picture,overlay]
    \ifportada
      \node at (current page.center)
      {\includegraphics[width=\paperwidth,height=\paperheight]{backgrounds/UISEK-portada.pdf}};
    \else
      \node at (current page.center)
      {\includegraphics[width=\paperwidth,height=\paperheight]{backgrounds/UISEK-plantilla.pdf}};
    \fi
  \end{tikzpicture}
}

\usepackage{booktabs}

%Resaltado de código
\usepackage{minted}
\usemintedstyle{colorful} % o 'colorful', 'monokai', 'borland', etc.

\setminted{
  %frame=lines,
  %framesep=2mm,
  baselinestretch=1.2,
  bgcolor=LightGray,
  fontsize=\small,
  breaklines=true,
  breakanywhere=true
}


\usepackage{csquotes} %cambiamos a este lugar cs quotes por si está antes de minted da problemas

%Bibliografía

%Bibliografía con Normas APA
%    \usepackage[backend=biber, style=apa]{biblatex}
%    \DeclareLanguageMapping{english}{english-apa} %\addbibresource{referencias.bib}
% útil para compactar las referencias
%{biblatex}
%    \setlength\bibitemsep{1.5ex}
%    \renewcommand*{\bibfont}{\small}

\begin{document}
% --- Portada sin número ---
\thispagestyle{empty}

% ---- aquí tu contenido de portada ----
\vspace*{9cm}
\begin{center}
{\fontsize{18}{22}\selectfont
\textbf{Maestría en bioinformática} \\
\textbf{Principios y Lógica de Programación} \\
\textbf{Docente: Ing. María Gabriela Echeverría}
}

\vspace{3cm}

{\fontsize{24}{28}\selectfont \textcolor{AMCblue}{\textbf{Actividad 1: Creación del directorio de trabajo}}}
\vspace*{3cm} % 3 cm más abajo

{\fontsize{20}{24}\selectfont \textbf{Estudiante: Christian Robles}}

\vspace*{3cm} % 2 cm más abajo
{\fontsize{12}{14}\selectfont Fecha de entrega: 20 de Enero de 2026}
\end{center}

\newpage

% <<< DESPUÉS de pasar la portada apagamos la bandera >>>
\portadafalse
\setcounter{page}{1} % reiniciamos el contador de páginas

\section{Introducción}
Esta actividad busca aplicar conceptos de organización en un \textit{espacio de trabajo} dentro del sistema operativo \textbf{GNU/Linux}, como la creación de \textit{directorios} y \textit{archivos}, además de aprender a movernos dentro de ellos de manera práctica. La tarea la llevé a cabo en \textbf{Debian 13 Trixie}, utilizando el entorno de escritorio \textbf{XFCE} y el gestor de archivos \textbf{Thunar}.

\section{Creación del directorio de trabajo}

Para comenzar el ejercicio, lo primero fue crear el directorio raíz del proyecto.
Este directorio sigue la convención propuesta \texttt{User\_project\_year\_publication}, en mi caso:

\begin{minted}{bash}
mkdir -v Robles_PhageOmics_2026_Science
cd Robles_PhageOmics_2026_Science
\end{minted}

El comando \texttt{mkdir} sirve para crear directorios.
La opción \texttt{-v} (verbose) muestra un mensaje del directorio que se creo, lo cual ayuda a verificar que se hizo correctamente.
Luego usamos \texttt{cd} para entrar al nuevo directorio y empezar a organizar la estructura interna.
\begin{figure}[htb]
    \centering
    \includegraphics[width=1\textwidth]{figuras/figura1.png}
    \caption{Creación del directorio principal del proyecto y de subdirectorios con el comando mkdir, creación de archivos inicales con touch.}
    \label{fig:figura1}
\end{figure}

\section{Creación de subdirectorios y archivos}

Una vez dentro del directorio principal, organicé la estructura básica del proyecto. Como se muestra en la \autoref{fig:figura1}.\\
Para ello, ejecuté los siguientes comandos:

\begin{minted}{bash}
mkdir -v analysis code data tools submission ; touch README LICENSE
\end{minted}

El comando \texttt{mkdir} permite crear varios directorios a la vez, en este caso:
\texttt{analysis}, \texttt{code}, \texttt{data}, \texttt{tools} y \texttt{submission}.
El comando \texttt{touch} se usó para generar dos archivos vacíos: \texttt{README} y \texttt{LICENSE}.

\section{Creación de subdirectorios adicionales}

Dentro del directorio \texttt{data}, añadí dos carpetas raw y processed que serán útiles para organizar los datos del proyecto. Ver la \autoref{fig:figura1}.

\begin{minted}{bash}
cd data
mkdir -v raw processed
\end{minted}

%De esta manera, los archivos originales se guardarán en \texttt{raw}, mientras que los resultados ya trabajados se almacenarán en \texttt{processed}.

Luego, dentro del directorio \texttt{submission}, creé dos directorios version1 y version2:

\begin{minted}{bash}
cd ../submission
mkdir -v version1 version2
\end{minted}

%Esto permite mantener un control claro de las distintas versiones que se preparen para enviar a una revista o conferencia.

\section{Verificación de la estructura con \texttt{tree}}

Después de crear todos los directorios y archivos, verifiqué que la estructura estuviera organizada correctamente.
Para esto utilicé el comando:

\begin{minted}{bash}
tree ~/Robles_PhageOmics_2026_Science/
\end{minted}

\begin{figure}[htb]
    \centering
    \includegraphics[width=1\textwidth]{figuras/figura2_tree1.png}
    \caption{Confirmación de la estructura de directorios con el comando \texttt{tree}}
    \label{fig:figura2}
\end{figure}


El comando \texttt{tree} muestra la estructura de carpetas y archivos en forma jerárquica.
Esto permite confirmar de manera visual que los directorios se crearon de forma correcta de acuardo al esquema planteado.
%\texttt{analysis}, \texttt{code}, \texttt{data}, \texttt{tools}, \texttt{submission}, y los archivos \texttt{README} y \texttt{LICENSE} se encuentran en el lugar correcto.
%Además, se puede ver que dentro de \texttt{data} están las carpetas \texttt{raw} y \texttt{processed}, y dentro de \texttt{submission} las carpetas \texttt{version1} y \texttt{version2}.

\begin{figure}[htb]
    \centering
    \includegraphics[width=0.9\textwidth]{figuras/figura4_thunar.png}
    \caption{Estructura de directorios visualizados en el gestor de archivos Thunar.}
    \label{fig:figura3}
\end{figure}

\section{Listado de inodos y permisos}

Para revisar las características de cada archivo y directorio, utilicé el siguiente comando:

\begin{minted}{bash}
ls -Rlhi
\end{minted}

\begin{figure}[htb]
    \centering
    \includegraphics[width=0.9\textwidth]{figuras/figura3_permisos_inodos.png}
    \caption{Inodos y permisos de los archivos del proyecto.}
    \label{fig:figura4}
\end{figure}

Este comando combina varias opciones:
\begin{enumerate}
    \item \texttt{R} → listado recursivo, muestra todo el contenido de subdirectorios.
    \item \texttt{l} → formato detallado, incluye permisos, propietario y tamaño.
    \item \texttt{h} → tamaños en formato legible (KB, MB).
    \item \texttt{i} → muestra el número de inodo de cada archivo o carpeta.
\end{enumerate}


De esta manera pude verificar tanto la estructura como los inodos y permisos de cada elemento del proyecto. Ver \autoref{fig:figura4}

\section{Cambio de permisos del archivo README}

El archivo \texttt{README} debemos otorgarle permisos de lectura y escritura para el usuario, y solo lectura para grupo y otros.
Para ello, ejecuté los comandos:

\begin{minted}{bash}
chmod 644 README
ls -lh README
\end{minted}

\begin{figure}[htb]
    \centering
    \includegraphics[width=1\textwidth]{figuras/figura5_readme_.png}
    \caption{Cambio de permisos del archivo README con \texttt{chmod}}
    \label{fig:figura5}
\end{figure}


El comando \texttt{chmod 644} asigna los permisos de la siguiente forma:
\begin{itemize}
    \item\texttt{6} → usuario: lectura y escritura.
    \item \texttt{4} → grupo: solo lectura.
    \item \texttt{4} → otros: solo lectura.
\end{itemize}

Finalmente, con \texttt{ls -lh README} confirmé que los permisos se aplicaron correctamente, mostrando \texttt{-rw-r--r--}.

\section{Conclusiones}

En esta actividad organizamos un directorio de trabajo para el proyecto, creando subcarpetas que nos ayudan a mantener todo en orden y encontrar rápido nuestros archivos.

Se usaron comandos básicos como \texttt{mkdir}, \texttt{touch}, \texttt{tree}, \texttt{ls} y \texttt{chmod} para construir la estructura de nuestro espacio de trabajo, revisarla y ajustar permisos.
Esta organización facilita que cualquier persona entienda el proyecto y refuerza la importancia de mantener el orden desde el inicio.


\section{Bibliografía}
%\nocite{*}
%\printbibliography[heading=none]

%\newpage
%\section*{Anexos}

\end{document}
